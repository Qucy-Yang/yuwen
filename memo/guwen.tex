% 2019-09-10 四年级开始学校开始文言文内容
% 需要对这方面进行基本的整理

\section{文言文整理}

  \subsection{字词}

  \begin{tasks}[style=enumerate](1)
     \task 矣

     为中国古代文言助词,用于句末,与“了”相同。


     \epigraph{骨已尽矣,而两狼之并驱如故。}{《聊斋志异·狼三则》}

     \task 曰

     说的意思。比如在《论语》中有大量的子曰:

     \epigraph{子曰:“学而时习之,不亦说乎?有朋自远方来,不亦乐乎? ”}{《论语》}

     \task 昔

     具有``以前,过去''的意思, 和“故”意思相近。

     \epigraph{昔孟母, 择邻处, 子不学, 断机杼。}{《三字经》}

     \epigraph{昔人已乘黄鹤去,此地空余黄鹤楼。
黄鹤一去不复返,白云千载空悠悠。}{《黄鹤楼》}

     \task 善

     在古文中有表示同意赞同的意思

     \epigraph{王曰:“善。” 乃下令。}{《邹忌讽齐王纳谏》}


  \end{tasks}


\section{文言典故}

   \subsection{赵\xpinyin{抃}{bian4}与``一琴一鹤''}

   形容行装简少,也比喻为官清廉。

   宋·沈括《梦溪笔谈》卷九:
   \begin{quotation}
      赵阅道\marginpar{\footnotesize{赵抃字阅道}}为成都转运史,出行部内,唯携一琴一鹤,坐则看鹤鼓琴。
    \end{quotation}


   《宋史·赵抃传》:
     \begin{quotation}
        帝曰:``闻卿匹马入蜀,以一琴一鹤自随;为政简易,亦称是乎!''
     \end{quotation}