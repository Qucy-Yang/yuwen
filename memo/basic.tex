% 基础内容,字词的整理

\section{易错字词}

\begin{tasks}[style=enumerate](2)
    \task 接着,救援, 温暖
    \task 柳树,欢迎, 昂首挺胸
    \task 彻底,倾盆大雨
    \task 萤火虫,荧光屏
    \task 暮气沉沉,幕布,落下帷幕
    \task 浪费,废除,残废
\end{tasks}

  \subsection{形状类似字词}
非常相似在几个汉字 - 戌 戍 戊 戒 戎

这5个字由于形体接近而及其容易混淆。为简化记忆,可以使用儿歌方式来记忆:

  \begin{center}横\xpinyin{戌}{xu1}点\xpinyin{戍}{shu4}\xpinyin{戊}{wu4}中空,\xpinyin{戒}{jie4}字去竖就念\xpinyin{戎}{rong2}\end{center}

  戊戌分别是天干(10个序号)地支(12生肖)。

  戍表示军队守卫,所以常见词有"卫戍、戍边、戍守"等。

  \subsection{形状吓人字词}
饕餮 耄耋

  \subsection{躁和燥} \marginpar{噪音}
按《说文解字》的说法:"躁,疾也"

所以躁通常表示人的性格、情绪等状态。

而燥与火、热有关。

\begin{question}
  暴\blank[width=1cm]{} \hspace{1cm} 干\blank[width=1cm]{}\hspace{1cm}  狂\blank[width=1cm]{}不安 \hspace{1cm}\blank[width=1cm]{}热

  \begin{tasks}(2)
    \task 躁  \task 燥
  \end{tasks}

\end{question}
\begin{solution}
  暴躁, 干燥, 狂躁不安, 燥热
\end{solution}

  \subsection{墓幕暮}

\begin{question}
选择正确的字词:

\begin{itemize}
  \item 我在\blank[width=0.5cm]{}色渐合的傍晚回到了家乡。
  \item 夜\blank[width=0.5cm]{}降临,星光灯火交相辉映,四处流光溢彩,如满天尘星,极为瑰丽。
  \item 久已有志于改革的王安石,受命执政,生气勃勃,但朝廷却是\blank[width=0.5cm]{}气沉沉。
  \item \blank*[width=0.5cm]{}色四合,华灯初上。
  \item 这次人民代表大会顺利落下帷\blank[width=0.5cm]{}。
\end{itemize}

 \begin{tasks}(2)
  \task 幕  \task 暮
 \end{tasks}

\end{question}
\begin{solution}
\begin{itemize}
  \item 我在\blank[width=0.5cm]{暮}色渐合的傍晚回到了家乡。
  \item 夜\blank[width=0.5cm]{幕}降临,星光灯火交相辉映,四处流光溢彩,如满天尘星,极为瑰丽。
  \item 久已有志于改革的王安石,受命执政,生气勃勃,但朝廷却是\blank[width=0.5cm]{暮}气沉沉。
  \item \blank*[width=0.5cm]{暮}色四合,华灯初上。
  \item 这次人民代表大会顺利落下帷\blank[width=0.5cm]{幕}。
\end{itemize}
\end{solution}

   \subsection{筹畴}

\begin{itemize}
  \item 畴 - 有田旁,本和田地有关,引申为类别,范围,领域意思
  \item 筹 - 有竹旁,本指计数用的竹签,引申为算计、策划等含义,它没有"范围"的意思
\end{itemize}

\begin{question}
判断下面句子是否有错误:

\begin{tasks}(1)
  \task 汉字属于表意文字的范筹。
  \task 他们在畴划一套完美的方案。
 \end{tasks}
\end{question}
\begin{solution}
  \begin{tasks}(2)
    \task 范畴。
    \task 筹划。
   \end{tasks}
\end{solution}

\section{易读错字词}
\begin{tasks}[style=enumerate](2)
    \task 友\xpinyinsetup{vsep=1.5em,ratio=0.9}\xpinyin{谊}{yi4}
    \task 造\xpinyinsetup{vsep=1.5em,ratio=0.9}\xpinyin{诣}{yi4}
    \task \xpinyinsetup{vsep=1.5em,ratio=0.9}\xpinyin{悖}{bei4}论
    \task 鳞次\xpinyinsetup{vsep=1.5em,ratio=0.9}\xpinyin{栉}{zhi4}比
    \task \xpinyinsetup{vsep=1.5em,ratio=0.9}\xpinyin{栉}{zhi4}风沐雨
\end{tasks}



\section{书面用语}

  \subsection{基础词语}

\begin{tasks}[style=enumerate](2)
    \task 精神矍铄  \index{难读字!矍铄}
    \task 佝偻      \index{难读字!佝偻}
    \task 罄竹难书, 售罄 \index{难读字!罄竹难书}
    \task 赝品      \index{难读字!赝品}
\end{tasks}

  \subsection{暌违} \marginpar{\pinyin{kui2 wei2}}
旧时书信用语,表示分离,不在一起的意思。

示例:

\begin{quotation}
  阳光高照,清风徐来,红墙环抱,松柏积翠,默然伫立\marginpar{\xpinyin[ratio={.7}]{伫}{zhu4}: \scriptsize{长时间站立}}在900多年前由宋高宗赐书的“龙光书院”匾额下,我不敢贸然前行了,一种暌违之感油然而生。
\end{quotation}

发散思维: \marginpar{众目睽睽}

\begin{tasks}[style=itemize](1)
   \task \xpinyin[ratio={.6}]{暌}{kui2}的另一个类似汉字是睽。
   \task \xpinyin[ratio={.6}]{癸}{gui3}为天干第十位
\end{tasks}

%  \task \xpinyin[ratio={.6}]{暌}{kui2}的另一个类似汉字是睽。\marginpar{众目睽睽}
%  \task \xpinyin[ratio={.6}]{癸}{gui3}为天干第十位
%\end{tasks}



\section{作文用语}

\begin{tasks}[style=enumerate](2)
    \task 恍然大悟  \index{作文用语!恍然大悟}
    \task 小心翼翼      \index{作文用语!小心翼翼}
    \task 昂首挺胸  \index{作文用语!昂首挺胸}
    \task 窃窃私语  \index{作文用语!窃窃私语}
    \task 猝不及防   \index{作文用语!猝不及防}
\end{tasks}